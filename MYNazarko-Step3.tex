\documentclass[12pt,pdf,hyperref={unicode}]{beamer}
%\usetheme{boxes}
\beamertemplatenavigationsymbolsempty
\setbeamertemplate{footline}[page number]
% Set it for the internal PhD thesis defence to reduce number of slides
%\setbeamersize{text margin left=0.5em, text margin right=0.5em}

\usepackage[utf8]{inputenc}
\usepackage[english, russian]{babel}
\usepackage{bm}
\usepackage{multirow}
\usepackage{ragged2e}
\usepackage{indentfirst}
\usepackage{multicol}
\usepackage{subfig}
\usepackage{amsmath,amssymb}
\usepackage{enumerate}
\usepackage{mathtools}
\usepackage{comment}
\usepackage[all]{xy}
\usepackage{tikz}
\usetikzlibrary{positioning,arrows}
\tikzstyle{name} = [parameters]
\definecolor{name}{rgb}{0.5,0.5,0.5}

%\usepackage{caption}
%\captionsetup{skip=0pt,belowskip=0pt}

%\newtheorem{theorem}{Theorem}
%\newtheorem{statement}{Statement}
%\newtheorem{definition}{Definition}

% colors
\definecolor{darkgreen}{rgb}{0.0, 0.2, 0.13}
\definecolor{darkcyan}{rgb}{0.0, 0.55, 0.55}
%\AtBeginEnvironment{figure}{\setcounter{subfigure}{0}}
%\captionsetup[subfloat]{labelformat=empty}

%----------------------------------------------------------------------------------------------------------

\title{Тэгирование с помощью LLM и ранжирование генеративных картинок}
%\author{Name Surname}
%\institute[]{}
%\date{2023}

%---------------------------------------------------------------------------------------------------------
\begin{document}
%\begin{frame}
%\titlepage
%\end{frame}
\setcounter{page}{2}%remove here for the title
%----------------------------------------------------------------------------------------------------------
%\section{Please do not use sectioning in the presentations}

\begin{frame}{Тэгирование генеративных картинок с помощью VLM }
\tiny % Уменьшаем шрифт для текущего слайда

  \textbf{Цель}: построить модель, которая обеспечит разнообразие ленты через эффективное тэгирование, генерируемых на основе текстовых промптов и сгенерирвоанного изображения.
  
  \begin{block}{Задачи}
    \begin{enumerate}[1.]
      \item Построить модель выделения ключевых сущностей из картинок и текстовых промптов.
      \item Создать систему ранжирования изображений на основе полученных тегов для улучшения разнообразия ленты.
    \end{enumerate}
  \end{block}

  \begin{block}{Исследуемая проблема}
    \begin{enumerate}[1.]
      \item Обеспечение единства и обобщенности тегов без декоративных и несущественных слов. Преобразование специфических обозначений в более общие теги (например, «котенок» в «кот»).
    \end{enumerate}
  \end{block}

  \begin{block}{Метод решения}
    Алгоритм тэгирования картинок и промптов основан на применении визуальноязыковой модели для выделения ключевых сущностей и гиперонимов. Это позволит игнорировать стилистические изменения и будет способствовать созданию обобщенного представления тегов.
  \end{block}

\end{frame}

%----------------------------------------------------------------------------------------------------------
\begin{frame}{Как это выглядит сейчас%
\footnote{\textit{М.\,Ю.~Назарько.}  Кадр c трудовых будней~// Работа, 2024.}}

  Типичная лента, до внедренения подобной модели
  \begin{columns}
    \begin{column}{0.3\textwidth}
      \begin{enumerate}[1)]
        \item Что происходило с персональными рекомендациями на 23 февраля 
      \end{enumerate}
    \end{column}
    \begin{column}{0.7\textwidth}
      \tiny 
     \includegraphics[width=0.45\textwidth]{image.png}      
    \end{column}
  \end{columns}
  \bigskip
\end{frame}
\end{document}
