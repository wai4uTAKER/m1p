\documentclass[12pt]{article}
\usepackage[utf8]{inputenc}
\usepackage[T1]{fontenc}
\usepackage{amsmath,amsfonts,amssymb}
\usepackage{graphicx}
\usepackage{a4wide}
\title{Reconstructed abstract of the paper ``GRAND: Graph Neural Diffusion''}
%\author{not specified, not necessary here}
\date{}
\begin{document}
\maketitle

\begin{abstract}
Graph Neural Diffusion (GRAND) reinterprets deep learning on graphs as a continuous diffusion process, viewing Graph Neural Networks as discretizations of partial differential equations. This approach develops diverse GNN architectures by varying temporal and spatial operator discretization. GRAND addresses issues like limited depth, over-smoothing, and bottlenecks. We present linear and nonlinear variants, explore numerical schemes, and provide stability analysis. GRAND shows competitive performance on benchmarks, offering a robust, flexible approach to graph learning with broad applications.
\end{abstract}
\paragraph{Keywords:} Graph Neural Networks, Partial Differential Equations, Diffusion processes 

\paragraph{Highlights:}
\begin{enumerate}
\item Principled approach to developing diverse GNN architectures
\item Addresses common GNN issues: limited depth, over-smoothing, and bottlenecks
\item Novel interpretation of GNNs as discretizations of continuous diffusion processes
\end{enumerate}

\section{Introduction}
The paper "GRAND: Graph Neural Diffusion"~\cite{unknown} was selected for its innovative approach to graph representation learning. It introduces GRAND, a framework that reinterprets Graph Neural Networks as discretizations of partial differential equations. This novel perspective addresses persistent challenges in GNNs, such as limited depth and over-smoothing, while providing a principled method for designing diverse architectures. The work's theoretical foundations, combined with competitive performance on benchmarks, demonstrate its potential to significantly impact the field of graph learning and its applications across various domains.
%\begin{figure}
%\includegraphics[scale=0.35]{SVD_derint}
%\caption{A rigorous description of what the reader sees on the plot and the consequences of the shown result}
%\end{figure}

\bibliographystyle{unsrt}
\bibliography{Name-theArt}
\end{document}